% 参考文献

% 参考文献开始
\unnumchapter{参考文献使用规则}
\renewcommand{\thechapter}{参考文献}

% 设置参考文献字号为 5 号
\renewcommand*{\bibfont}{\zihao{5}}
% 设置参考文献各个项目之间的垂直距离为 0
\setlength{\bibitemsep}{0ex}
\setlength{\bibnamesep}{0ex}
\setlength{\bibinitsep}{0ex}
% 设置单倍行距
\renewcommand{\baselinestretch}{1.2}
% 设置参考文献顺序标签 `[1]` 与文献内容 `作者. 文献标题...` 的间距
\setlength{\biblabelsep}{0.5mm}
% 设置参考文献后文缩进为 0(与 Word 模板保持一致)
\renewcommand{\itemcmd}{
  \addvspace{\bibitemsep} % 恢复 \bibitemsep 的作用
  \mkgbnumlabel{\printfield{labelnumber}}
  \hspace{\biblabelsep}}
% 删除默认的「参考文献 / Reference」标题,使用上面定义的 section 标题

\textcolor{blue}{参考文献书写规范}

\textcolor{blue}{参考国家标准《信息与文献参考文献著录规则》【GB/T 7714—2015】,参考文献书写规范如下:}

\textcolor{blue}{\textbf{1. 文献类型和标识代码}}

\textcolor{blue}{普通图书:M}\qquad\textcolor{blue}{会议录:C}\qquad\textcolor{blue}{汇编:G}\qquad\textcolor{blue}{报纸:N}

\textcolor{blue}{期刊:J}\qquad\textcolor{blue}{学位论文:D}\qquad\textcolor{blue}{报告:R}\qquad\textcolor{blue}{标准:S}

\textcolor{blue}{专利:P}\qquad\textcolor{blue}{数据库:DB}\qquad\textcolor{blue}{计算机程序:CP}\qquad\textcolor{blue}{电子公告:EB}

\textcolor{blue}{档案:A}\qquad\textcolor{blue}{舆图:CM}\qquad\textcolor{blue}{数据集:DS}\qquad\textcolor{blue}{其他:Z}

\textcolor{blue}{\textbf{2. 不同类别文献书写规范要求}}

\textcolor{blue}{\textbf{期刊}}

\noindent\textcolor{blue}{[序号]主要责任者. 文献题名[J]. 刊名, 出版年份, 卷号(期号): 起止页码. }


\textcolor{blue}{\textbf{普通图书}}

\noindent\textcolor{blue}{[序号]主要责任者. 文献题名[M]. 出版地: 出版者, 出版年. 起止页码. }
\cite{Raymer1992Aircraft}


\textcolor{blue}{\textbf{会议论文集}}

\noindent\textcolor{blue}{[序号]析出责任者. 析出题名[A]. 见(英文用In): 主编. 论文集名[C]. (供选择项: 会议名, 会址, 开会年)出版地: 出版者, 出版年. 起止页码. }
\cite{sunpinyi}


\textcolor{blue}{\textbf{专著中析出的文献}}

\noindent\textcolor{blue}{[序号]析出责任者. 析出题名[A]. 见(英文用In): 专著责任者. 书名[M]. 出版地: 出版者, 出版年.起止页码. }
\cite{luoyun}


\textcolor{blue}{\textbf{学位论文}}

\noindent\textcolor{blue}{[序号]主要责任者. 文献题名[D]. 保存地: 保存单位, 年份. }
\cite{zhanghesheng,Sobieski}


\textcolor{blue}{\textbf{报告}}

\noindent\textcolor{blue}{[序号]主要责任者. 文献题名[R]. 报告地: 报告会主办单位, 年份. }
\cite{fengxiqiao,Sobieszczanski}


\textcolor{blue}{\textbf{专利文献}}

\noindent\textcolor{blue}{[序号]专利所有者. 专利题名[P]. 专利国别: 专利号, 发布日期. }
\cite{jiangxizhou}


\textcolor{blue}{\textbf{国际、国家标准}}

\noindent\textcolor{blue}{[序号]标准代号. 标准名称[S]. 出版地: 出版者, 出版年. }
\cite{GB/T16159—1996}


\textcolor{blue}{\textbf{报纸文章}}

\noindent\textcolor{blue}{[序号]主要责任者. 文献题名[N]. 报纸名, 出版年, 月(日): 版次. }
\cite{xiexide}


\textcolor{blue}{\textbf{电子文献}}

\noindent\textcolor{blue}{[序号]主要责任者. 电子文献题名[文献类型/载体类型]. 电子文献的出版或可获得地址(电子文献地址用文字表述), 发表或更新日期/引用日期(任选). }
\cite{yaoboyuan}


\textcolor{blue}{关于参考文献的未尽事项可参考国家标准《信息与文献参考文献著录规则》(GB/T 7714—2015)}

%输出所有的参考文献
%\printbibliography[heading=none]
