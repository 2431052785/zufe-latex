% 原创性声明页
% 无特殊要求,不用修改

\fancypagestyle{originality}{
  % 页眉高度
  \setlength{\headheight}{10pt}

  % 页眉和页脚(页码)的格式设定
  \fancyhf{}
  \fancyhead[]{}

  % 页眉分割线稍微粗一些
  \renewcommand{\headrulewidth}{0pt}
}

\pagestyle{originality}
% \topskip=0pt

% % 圆形数字编号定义
% \newcommand{\circled}[2][]{\tikz[baseline=(char.base)]
%   {\node[shape = circle, draw, inner sep = 1pt]
%   (char) {\phantom{\ifblank{#1}{#2}{#1}}};
%   \node at (char.center) {\makebox[0pt][c]{#2}};}}
% \robustify{\circled}

% 设置行间距
\setlength{\parskip}{0.4em}
\renewcommand{\baselinestretch}{1.41}

% 顶部空白
\vspace*{-6mm}

% 原创性声明部分
\begin{center}
  \heiti\zihao{2}\textmd{声明及论文使用的授权}
\end{center}

\vspace{10mm}


% 本部分字号为小三
\zihao{-3}

本人郑重声明所呈交的论文是我个人在导师的指导下独立完成的。除了文中特别加以标注和致谢的地方外,论文中不包含其他人已经发表或撰写的研究成果。

\vspace{15mm}

\begin{flushright}
  论文作者签名:\hspace{75mm}年\hspace{8mm}月\hspace{8mm}日
\end{flushright}

\vspace{40mm}

% 使用授权声明部分

\zihao{-3}

本人同意浙江财经大学有关保留使用学位论文的规定,即:学校有权保留送交论文的复印件,允许论文被查阅和借阅;学校可以上网公布全部内容,可以采用影印、缩印或其他复制手段保存论文。

\vspace*{15mm}

\begin{flushright}
  论文作者签名:\hspace{75mm}年\hspace{8mm}月\hspace{8mm}日
\end{flushright}

\newpage
