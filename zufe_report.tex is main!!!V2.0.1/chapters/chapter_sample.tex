% !TeX root = ../zufe_report.tex
\setlength{\baselineskip}{20pt}
% 样例章节

% 章和引用示例-------------------------------------------------------------------------------------
\chapter{一级题目}

\section{二级题目}

正文······\cite{GB/T16159—1996} % 引用示例

\subsection{三级题目}

正文······\cite{Sobieski}
% 章和引用示例-------------------------------------------------------------------------------------


% 表示例----------------------------------------------------------------------------------------
\section{\textcolor{blue}{\underline{\underline{表-示例}}}}

\textcolor{blue}{{自动生成LaTeX表工具: \url{https://www.tablesgenerator.com/}}}

\begin{table}[htbp]
  \linespread{1.5}
  \zihao{5}
  \songti
  \centering
  \caption{物流的概念和范围}\label{物流的概念和范围}
  \begin{tabular}{|l|l|}
  \hline
  \multicolumn{1}{|c|}{本 质} & \multicolumn{1}{c|}{过  程}  \\ \hline
  途径或方法                     & 规划、实施、控制                   \\ \hline
  目标                        & 效率、成本效益                    \\ \hline
  活动或作业                     & 流动与储存                      \\ \hline
  处理对象                      & 原材料、在制品、产成品、相关信息           \\ \hline
  范围                        & 从原点(供应商)到终点(最终顾客)          \\ \hline
  目的或目标                     & 适应顾客的需求(产品、功能、数量、质量、时间、价格) \\ \hline
  \end{tabular}
\end{table}

 \begin{table}[htbp]
   \linespread{1.5}
   \zihao{5}
   \songti
   \centering
   \caption{统计表}\label{统计表}
   \begin{tabular}{|l|l|l|l|l|}
   \hline
   产品  & 产量    & 销量    & 产值   & 比重    \\ \hline
   手机  & 11000 & 10000 & 500  & 50\%  \\ \hline
   电视机 & 5500  & 5000  & 220  & 22\%  \\ \hline
   计算机 & 1100  & 1000  & 280  & 28\%  \\ \hline
   合计  & 17600 & 16000 & 1000 & 100\% \\ \hline
   \end{tabular}
 \end{table}

\begin{table}[htbp]
  \linespread{1.5}
  \zihao{5}
  \songti
  \centering
  \caption{统计表}\label{统计表}
  % Please add the following required packages to your document preamble:
  % \usepackage{multirow}
  \begin{tabular}{|l|l|l|l|l|}
  \hline
  年度                    & 产品  & 产量    & 销量    & 产值  \\ \hline
  \multirow{2}{*}{2004} & 手机  & 11000 & 10000 & 500 \\ \cline{2-5} 
                      & 计算机 & 1100  & 1000  & 280 \\ \hline
  \multirow{2}{*}{2005} & 手机  & 16000 & 13000 & 550 \\ \cline{2-5} 
                      & 计算机 & 2100  & 1500  & 320 \\ \hline
  \end{tabular}
\end{table}
% 表示例----------------------------------------------------------------------------------------

\section{\textcolor{blue}{\underline{\underline{表格-示例}}}}


表应具有自明性。为使表格简洁易读,尽可能采用三线表,如表~\ref{tab:three-line}。
三条线可以使用 booktabs 宏包提供的命令生成。

\begin{table}[htbp]
	\centering
	\caption{三线表示例}
	\begin{tabular}{ll}
		\toprule
		文件名          & 描述                         \\
		\midrule
		zufereport.dtx   & 模板的源文件,包括文档和注释 \\
		zufereport.cls   & 模板文件                     \\
		zufereport-*.bst & BibTeX 参考文献表样式文件    \\
		\bottomrule
	\end{tabular}
	\label{tab:three-line}
\end{table}

表格如果有附注,尤其是需要在表格中进行标注时,可以使用 threeparttable 宏包。
研究生要求使用英文小写字母 a、b、c……顺序编号,本科生使用圈码 ①、②、③……编号。

\begin{table}[htbp]
	\centering
	\begin{threeparttable}[c]
		\caption{带附注的表格示例}
		\label{tab:three-part-table}
		\begin{tabular}{ll}
			\toprule
			文件名                 & 描述                         \\
			\midrule
			zufereport.dtx\tnote{a} & 模板的源文件,包括文档和注释 \\
			zufereport.cls\tnote{b} & 模板文件                     \\
			zufereport-*.bst        & BibTeX 参考文献表样式文件    \\
			\bottomrule
		\end{tabular}
		\begin{tablenotes}
			\item [a] 可以通过 xelatex 编译生成模板的使用说明文档;
			使用 xetex 编译 \file{zufereport.ins} 时则会从 \file{.dtx} 中去除掉文档和注释,得到精简的 \file{.cls} 文件。
			\item [b] 更新模板时,一定要记得编译生成 \file{.cls} 文件,否则编译论文时载入的依然是旧版的模板。
		\end{tablenotes}
	\end{threeparttable}
\end{table}

如某个表需要转页接排,可以使用 longtable 宏包,需要在随后的各页上重复表的编号。
编号后跟表题(可省略)和“(续)”,置于表上方。续表均应重复表头。

\singlespacing
\begin{longtable}{cccc}
	\caption{跨页长表格的表题}
	\label{tab:longtable} \\
	\toprule
	表头 1 & 表头 2 & 表头 3 & 表头 4 \\
	\midrule
	\endfirsthead
	\caption*{续表~\thetable\quad 跨页长表格的表题} \\
	\toprule
	表头 1 & 表头 2 & 表头 3 & 表头 4 \\
	\midrule
	\endhead
	\bottomrule
	\endfoot
	Row 1  & & & \\
	Row 2  & & & \\
	Row 3  & & & \\
	Row 4  & & & \\
	Row 5  & & & \\
	Row 6  & & & \\
	Row 7  & & & \\
	Row 8  & & & \\
	Row 9  & & & \\
	Row 10 & & & \\
\end{longtable}

% 伪代码示例------------------------------------------------------------------------------------

\section{\textcolor{blue}{\underline{\underline{伪代码-示例}}}}

\textcolor{blue}{修改 algorithmic 之间的代码就可以实现论文伪代码,已经考虑了伪代码跨页问题}
\begin{breakablealgorithm}
        \caption{Calculate $y = x^n$}
        \begin{algorithmic}[1] %每行显示行号
            \Require $n \geq 0 \vee x \neq 0$ 
            \Ensure  $y = x^n$
            \State $y \gets 1$
            \If{$n < 0$}
            \State $ X \gets 1 / x$
            \State $N \gets -n$
            \Else
            \State $X \gets x$
            \State $N \gets n$
            \EndIf
            \While{ $N \neq 0$ }
            \If{ $N$ is even }
            \State $X \gets x \times x$
            \State $N \gets N / 2$
            \Else[$N$ is odd]
            \State $y \gets y \times X$
            \State $N \gets N - 1$
            \EndIf
            \EndWhile
     \end{algorithmic}
    \end{breakablealgorithm}
% 伪代码示例------------------------------------------------------------------------------------

% 代码块示例------------------------------------------------------------------------------------
\section{\textcolor{blue}{\underline{\underline{代码块-示例}}}}
\textcolor{blue}{只写了 C++, Python, Java 三种语言的格式}

% C++ 代码引入 + 文件引入
\lstinputlisting[
style = C++,
caption = {test.cpp},
label = {test.cpp}
]{./papperCode/test.cpp}

% Java 代码引入 + 行内引入
\begin{lstlisting}[
style = Java,
caption = {test.jar},
label = {test.jar}
]
public class HelloWorld {
    public static void main(String[] args){
        System.out.println("Hello World!");
    }
}
\end{lstlisting}

% Python 代码引入 + 文件引入
\lstinputlisting[
style = Python,
caption = {test.py},
label = {test.py}
]{./papperCode/test.py}
% 代码块示例------------------------------------------------------------------------------------