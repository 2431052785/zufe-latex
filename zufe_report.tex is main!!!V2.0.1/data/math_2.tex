% !TeX root = ../zufe_report.tex
\setlength{\baselineskip}{20pt}

\section{数学公式部分}
\subsection{数学符号和公式}
\subsubsection{标准}
中文论文的数学符号默认遵循 GB/T 3102.11—1993《物理科学和技术中使用的数学符号》
\footnote{原 GB 3102.11—1993,自 2017 年 3 月 23 日起,该标准转为推荐性标准。}。
该标准参照采纳 ISO 31-11:1992 \footnote{目前已更新为 ISO 80000-2:2019。},
但是与 \TeX{} 默认的美国数学学会(AMS)的符号习惯有所区别。
具体地来说主要有以下差异:
\begin{enumerate}
	\item 
	大写希腊字母默认`为斜体,如
	\begin{equation*}
		\varGamma \varDelta \varTheta \varLambda \varXi \varPi \varSigma \varUpsilon \varPhi \varPsi \varOmega.
	\end{equation*}
	\item 小于等于号和大于等于号使用倾斜的字形 $\le$、$\ge$。
	\item 积分号使用正体,比如 $\int$、$\oint$。
	\item
	偏微分符号 $\partial$ 使用正体。
	\item
	省略号按照中文的习惯固定居中,比如
	\begin{equation*}
		1, 2, \cdots, n \quad 1 + 2 + \cdots + n.
	\end{equation*}
	\item
	实部 $\operatorname{Re}$ 和虚部 $\operatorname{Im}$ 的字体使用罗马体。
\end{enumerate}




\subsubsection{TeX{}基础操作}
\begin{enumerate}
	\item TeX{}中的行内公式使用美元符号括起,行内公式示例:质能方程:$E=mc^2$
	\item 公式单独显示操作如下:
	
	
	自动编号
	\begin{equation}
		E=mc^2
	\end{equation}
	
	
	不自动编号
	\begin{equation*}
		E=mc^2
	\end{equation*}
	
	
	
	美元符号
	$$E=mc^2$$
	\item 上标,上标只有单个字符时使用“\^{}”符号,如:$E=mc^2$,否则将上标用大括号括起即可,如:$2^{x^{2}+y}$
	\item 下标,下标与上标相似,单个字符使用“\_{}”字符,如$a_1$,否则用大括号括起即可,如:$a_{100}$
	\item 分式:$frac{3}{4}$
	\item 根号:$\sqrt{2}$
	\item 矩阵: 矩阵使用matrix环境排版
	$$\begin{matrix}
		0 & 1 \\
		1 & 0
	\end{matrix}$$
	
	
	patrix环境用于在矩阵两端加小括号
	$$\begin{pmatrix}
		0 & 1 \\
		1 & 0
	\end{pmatrix}$$
	
	
	batrix环境用于在矩阵两端加中括号
	\begin{equation*}
		\begin{bmatrix}
			0 & 1 \\
			1 & 0
		\end{bmatrix}
	\end{equation*}
	
	Batrix环境用于在矩阵两端加大括号
	\begin{equation*}
		\begin{Bmatrix}
			0 & 1 \\
			1 & 0
		\end{Bmatrix}
	\end{equation*}
	
	vatrix环境用于在矩阵两端加竖线
	\begin{equation*}
		\begin{vmatrix}
			0 & 1 \\
			1 & 0
		\end{vmatrix}
	\end{equation*}
	
	Vatrix环境用于在矩阵两端加双竖线
	\begin{equation*}
		\begin{Vmatrix}
			0 & 1 \\
			1 & 0
		\end{Vmatrix}
	\end{equation*}
	
	矩阵中的省略号可用$\ddots,\vdots,\dots$实现
	\iffalse
	矩阵
	&为列分隔符,\\为行分隔符
	\fi
	\item 多行公式,多行公式用gather环境实现
	\begin{gather}
		E=mc^2\\
		E=mc^2
	\end{gather}
	不带编号:
	\begin{gather*}
		E=mc^2\\
		E=mc^2
	\end{gather*}
	单行不编号:使用notag命令
	\begin{gather}
		E=mc^2 \notag \\
		E=mc^2
	\end{gather}
	一个公式的多行编写:
	\begin{equation}
		\begin{split}
			E & =mc^2  \\
			& =mc^2
		\end{split}
	\end{equation}
	\iffalse
	一个公式的多行编写:&为对齐位置
	\fi
	\item 分段函数:
	\begin{equation}
		D(x)=\begin{cases}
			a, &0 \\
			b, &1
			%&为对齐位置
		\end{cases}
	\end{equation} 
\end{enumerate}

\subsubsection{数学公式}

数学公式可以使用 ${equation}$ 和 ${equation*}$ 环境。
注意数学公式的引用应前后带括号,通常使用 ${eqref}$ 命令,比如式\ref{eq:example}。
\iffalse
eqref需要使用标签方式
\fi
\begin{equation}
	\frac{1}{2 \uppi i} \int_\gamma f = \sum_{k=1}^m n(\gamma; a_k) R(f; a_k).
	\label{eq:example}
\end{equation}

多行公式尽可能在“=”处对齐,推荐使用 ${align}$ 环境。
\begin{align}
	a & = b + c + d + e \\
	& = f + g
\end{align}



\subsubsection{数学定理}

定理环境的格式可以使用 ${amsthm}$ 或者 ${ntheorem}$ 宏包配置。
用户在导言区载入这两者之一后,模板会自动配置 ${thoerem}$、${proof}$ 等环境。

\begin{theorem}[Lindeberg--Lévy 中心极限定理]
	设随机变量 $X_1,X_2, \dots, X_n$ 独立同分布, 且具有期望 $\mu$ 和有限的方差 $\sigma^2 \ne 0$,
	记 $\bar{X}_n = \frac{1}{n} \sum_{i+1}^n X_i$,则
	\begin{equation}
		\lim_{n \to \infty} P \left(\frac{\sqrt{n} \left( \bar{X}_n - \mu \right)}{\sigma} \le z \right) = \Phi(z),
	\end{equation}
	其中 $\Phi(z)$ 是标准正态分布的分布函数。
\end{theorem}
\begin{proof}
	Trivial.
\end{proof}
