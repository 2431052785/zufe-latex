\RequirePackage{CJKnumb}
\setlength{\parskip}{1em}
\renewcommand{\baselinestretch}{1.0}
%页面大小
\usepackage{geometry}
%首航缩进
\usepackage{indentfirst}
%各种下划线
\usepackage{ulem}
%加载图片
\usepackage{graphicx} 
%加载公式
\usepackage{amsmath}
\usepackage{amsthm}
%调整行间距
\usepackage{setspace}
%处理数学公式中和黑斜体
\usepackage{latexsym,bm}
%定义书写公式环境
\usepackage{cases}
%一些特殊符号例如:※√
\usepackage{pifont}
%文本和数学字符库
\usepackage{txfonts}
%手动设置行距
\usepackage{setspace}
%%%%%%%%%%% CJK下设置中文字体 %%%%%%%%%%%%%
\newcommand{\song}{\CJKfamily{song}}    % 宋体   (Windows自带simsun.ttf)
\newcommand{\fs}{\CJKfamily{fs}}        % 仿宋体 (Windows自带simfs.ttf)
\newcommand{\kai}{\CJKfamily{kai}}      % 楷体   (Windows自带simkai.ttf)
\newcommand{\hei}{\CJKfamily{hei}}      % 黑体   (Windows自带simhei.ttf)
\newcommand{\li}{\CJKfamily{li}}        % 隶书   (Windows自带simli.ttf)

%%%%%%%%%%%  设置字体大小 ,拼音代表字号%%%%%%%%%%%%%
\newcommand{\xiaochuhao}{\fontsize{36pt}{\baselineskip}\selectfont}
\newcommand{\xiaoerhao}{\fontsize{18pt}{\baselineskip}\selectfont}
\newcommand{\xiaosihao}{\fontsize{12pt}{\baselineskip}\selectfont}
\newcommand{\xiaowuhao}{\fontsize{9pt}{\baselineskip}\selectfont}
%三线表
\usepackage{threeparttable}

% 表格中支持跨行
\usepackage{multirow}

% 固定宽度的表格。
\usepackage{tabularx}

% 跨页表格
\usepackage{longtable}

%给字母加粗
\usepackage[T1]{fontenc}

% 量和单位
\usepackage{siunitx}

%三线表
\usepackage{booktabs}

%调用子图
\usepackage{subfigure}
%页眉页脚
\usepackage{fancyhdr}


% 定义所有的图片文件在 figures 子目录下
\graphicspath{{figures/}}
\newtheorem{theorem}{定理}
\usepackage{upgreek}
% 数学命令
\makeatletter
\newcommand\dif{%  % 微分符号
	\mathop{}\!%
	\ifthu@math@style@TeX
	d%
	\else
	\mathrm{d}%
	\fi
}
\makeatother

%新罗马字体
\newcommand{\romann}{\CJKfontspec{TIMES.TTF}\selectfont}
%中英文摘要
\newcommand{\enabstractname}{}
\newcommand{\cnabstractname}{}
\newenvironment{enabstract}{%
	\par\small
	\noindent\mbox{}\hfill{\bfseries\enabstractname}\hfill\mbox{}\par
	\vskip 2.5ex}{\par\vskip2.5ex}
\newenvironment{cnabstract}{%
	\par\small
	\noindent\mbox{}\hfill{\bfseries\cnabstractname}\hfill\mbox{}\par
	\vskip 2.5ex}{\par\vskip2.5ex}
{\tiny {\tiny }}
\usepackage{titletoc}
%目录加点
\usepackage[subfigure]{tocloft}      %必须这么写,否则会报错
%设置chapter条目的引导点间距
\renewcommand{\cftpartleader}{\cftdotfill{0.0001}}
\renewcommand{\cftsecleader}{\cftdotfill{0.0001}}
\renewcommand{\cftsubsecleader}{\cftdotfill{0.0001}}
%目录加点
\usepackage[subfigure]{tocloft}      %必须这么写,否则会报错
%设置chapter条目的引导点间距
\renewcommand{\cftpartleader}{\cftdotfill{0.0001}}
\renewcommand{\cftsecleader}{\cftdotfill{0.0001}}
\renewcommand{\cftsubsecleader}{\cftdotfill{0.0001}}

%设置目录格式
\usepackage{titletoc}
\titlecontents{part}[4em]
{\heiti\zihao{4}}
{\contentslabel{4em}}
{\hspace*{-4em}}
{\titlerule*[5pt]{.}\contentspage}
[\addvspace{0pt}]

\titlecontents{section}[2.5em]
{\vspace*{0ex}\songti\zihao{-4}}
{\contentslabel[\thecontentslabel]{2em}}
{\hspace*{-1em}}
{\titlerule*[5pt]{.}\contentspage}
[\addvspace{0ex}]

\titlecontents{subsection}[5.0em]
{\vspace*{1ex}\songti\zihao{-4}}
{\contentslabel[\thecontentslabel]{3.0em}}
{}
{\titlerule*[5pt]{.}\contentspage}
[\addvspace{0pt}]

\titlecontents{subsubsection}[5.0em]
{\vspace*{1ex}\songti\zihao{-4}}
{\contentslabel[\thecontentslabel]{3.0em}}
{}
{\titlerule*[5pt]{.}\contentspage}
[\addvspace{0pt}]