\chapter{表格与算法}

\section{表格}
自动生成LaTeX表工具: \url{https://www.tablesgenerator.com/}
\par
表应具有自明性。为使表格简洁易读,尽可能采用三线表,如表~\ref{tab:three-line}。
三条线可以使用 booktabs 宏包提供的命令生成。

\begin{table}[htbp]
	\centering
	\caption{三线表示例}
	\begin{tabular}{ll}
		\toprule
		文件名          & 描述                         \\
		\midrule
		zufereport.dtx   & 模板的源文件,包括文档和注释 \\
		zufereport.cls   & 模板文件                     \\
		zufereport-*.bst & BibTeX 参考文献表样式文件    \\
		\bottomrule
	\end{tabular}
	\label{tab:three-line}
\end{table}

表格如果有附注,尤其是需要在表格中进行标注时,可以使用 threeparttable 宏包。
研究生要求使用英文小写字母 a、b、c……顺序编号,本科生使用圈码 ①、②、③……编号。

\begin{table}[htbp]
	\centering
	\begin{threeparttable}[c]
		\caption{带附注的表格示例}
		\label{tab:three-part-table}
		\begin{tabular}{ll}
			\toprule
			文件名                 & 描述                         \\
			\midrule
			zufereport.dtx\tnote{a} & 模板的源文件,包括文档和注释 \\
			zufereport.cls\tnote{b} & 模板文件                     \\
			zufereport-*.bst        & BibTeX 参考文献表样式文件    \\
			\bottomrule
		\end{tabular}
		\begin{tablenotes}
			\item [a] 可以通过 xelatex 编译生成模板的使用说明文档;
			使用 xetex 编译 \file{zufereport.ins} 时则会从 \file{.dtx} 中去除掉文档和注释,得到精简的 \file{.cls} 文件。
			\item [b] 更新模板时,一定要记得编译生成 \file{.cls} 文件,否则编译论文时载入的依然是旧版的模板。
		\end{tablenotes}
	\end{threeparttable}
\end{table}

如某个表需要转页接排,可以使用 \usepackage{longtable} 宏包,需要在随后的各页上重复表的编号。
编号后跟表题(可省略)和“(续)”,置于表上方。续表均应重复表头。

\singlespacing
\begin{longtable}{cccc}
	\caption{跨页长表格的表题}
	\label{tab:longtable} \\
	\toprule
	表头 1 & 表头 2 & 表头 3 & 表头 4 \\
	\midrule
	\endfirsthead
	\caption*{续表~\thetable\quad 跨页长表格的表题} \\
	\toprule
	表头 1 & 表头 2 & 表头 3 & 表头 4 \\
	\midrule
	\endhead
	\bottomrule
	\endfoot
	Row 1  & & & \\
	Row 2  & & & \\
	Row 3  & & & \\
	Row 4  & & & \\
	Row 5  & & & \\
	Row 6  & & & \\
	Row 7  & & & \\
	Row 8  & & & \\
	Row 9  & & & \\
	Row 10 & & & \\
\end{longtable}


\section{伪代码}


算法环境可以使用 algorithms 或者 algorithm2e 宏包。

\renewcommand{\algorithmicrequire}{\textbf{Input:}\unskip}
\renewcommand{\algorithmicensure}{\textbf{Output:}\unskip}
\begin{breakablealgorithm}
	\caption{QuickSort 示例}
	\begin{algorithmic}[1] % 每行显示行号
		\Require $Array$数组,$n$数组大小
		\Ensure 排序后的数组
		\Function{QuickSort}{$Array, left, right$}
		\If {$left < right$}
		\State $pivot \gets$ \Call{Partition}{$Array, left, right$}
		\State \Call{QuickSort}{$Array, left, pivot-1$}
		\State \Call{QuickSort}{$Array, pivot+1, right$}
		\EndIf
		\EndFunction
		\State
		\Function{Partition}{$Array, left, right$}
		\State $pivot \gets Array[right]$
		\State $i \gets left-1$
		\For{$j \gets left$ \textbf{to} $right-1$}
		\If{$Array[j] < pivot$}
		\State $i \gets i+1$
		\State \Call{Swap}{$Array[i], Array[j]$}
		\EndIf
		\EndFor
		\State \Call{Swap}{$Array[i+1], Array[right]}$
		\State \Return{$i+1$}
		\EndFunction
		\State
		\Function{Swap}{$a, b$}
		\State $temp \gets a$
		\State $a \gets b$
		\State $b \gets temp$
		\EndFunction
	\end{algorithmic}

\end{breakablealgorithm}

\section{代码示例}
\lstinputlisting[style=C++, caption={示例 C++ 代码}, label={lst:cpp}]{Codes/test2.cpp}
\lstinputlisting[style=Python,caption={示例 Python 代码},label={lst:py}]{Codes/decision_tree.py}
\lstinputlisting[style=Java,caption={示例 Java 代码},label={lst:java}]{Codes/test.java}
