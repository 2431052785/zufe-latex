
% 下面是中文摘要 3-19行不用管
\begin{center}
	\heiti\zihao{3}\textmd{\thesisname}
	\vspace{2mm}
	\ifx\subtitle\undefined\else
	%如果有副标题
	\begin{spacing}{1.2}
		\sanhao\selectfont{\textmd{\kaishu{-----}\subtitle}}
	\end{spacing}
	\fi
\end{center}
%题目后空一行
\vspace{1em}

\setstretch{1.53}
\setlength{\parskip}{0em}

\textbf{\heiti\xiaosi{摘要:}}
%中文摘要----------------------------------------------------------
论文的摘要是对论文研究内容和成果的高度概括。
摘要应对论文所研究的问题及其研究目的进行描述,对研究方法和过程进行简单介绍,对研究成果和所得结论进行概括。
摘要应具有独立性和自明性,其内容应包含与论文全文同等量的主要信息。
使读者即使不阅读全文,通过摘要就能了解论文的总体内容和主要成果。
\par
论文摘要的书写应力求精确、简明。
切忌写成对论文书写内容进行提要的形式,尤其要避免“第 1 章……;第 2 章……;……”这种或类似的陈述方式。
\par
关键词是为了文献标引工作、用以表示全文主要内容信息的单词或术语。关键词不超过 5 个,每个关键词中间用分号分隔。
\par
\vspace{1em}
\textbf{\heiti\xiaosi{关键词:}}
%这里写你的关键词------------------------------------------------------
关键词1;关键词2;……
\vspace{15mm}
%如果没有英文标题,跳过
\ifx\enthesisname\undefined\else
%下面是英文摘要 37-52行不用管
\begin{spacing}{0.95}
	\centering
	\sanhao\textbf{\rmfamily{\enthesisname}}
	\vspace{2mm}
	\ifx\subtitle\undefined\else
	%如果有副标题
	\begin{spacing}{1.2}
		\sihao\selectfont{\textmd{\kaishu{-----}\rmfamily\textbf{\ensubtitle}}}
	\end{spacing}
	\fi
\end{spacing}
\vspace{0em}
\setstretch{1.53}
\setlength{\parskip}{0em}

\textbf{\rmfamily\xiaosi{Abstract: }}
% 英文摘要正文从这里开始--------------------------------------------------
An abstract of a dissertation is a summary and extraction of research work and contributions.
Included in an abstract should be description of research topic and research objective, brief introduction to methodology and research process, and summary of conclusion and contributions of the research.
An abstract should be characterized by independence and clarity and carry identical information with the dissertation.
It should be such that the general idea and major contributions of the dissertation are conveyed without reading the dissertation.
\par
An abstract should be concise and to the point.
It is a misunderstanding to make an abstract an outline of the dissertation and words “the first chapter”, “the second chapter” and the like should be avoided in the abstract.
\par

\vspace{1em}
\textbf{\rmfamily\xiaosi{Key words:}}
%这里写你的英文关键词----------------------------------------------------
keywords1;keywords2;...
\fi
