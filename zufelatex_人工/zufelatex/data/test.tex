\documentclass{ctexart}
\renewcommand{\baselinestretch}{1.5}
\setlength{\parskip}{1em}
\begin{document}
上市公司盈利能力评价

摘要:盈利能力是企业营销能力、获取现金能力、降低成本能力及规避风险能力等的综合体现,因此上市公司的盈利能力是市场投资者和经营者非常关注的热点问题,也是金融、投资领域中一个不容忽视的重要方面。本文从上市公司的当前盈利性、获现性和未来成长性三个方面着手,选取相应的财务指标,并采用层次分析法来客观、准确地确定各指标的相对权重,在此基础上构建上市公司盈利能力综合评价模型,并运用该模型对2009年上海证券交易所的5家酒业公司进行实证分析。

关键词:上市公司;盈利能力;评价指标;层次分析法


Evaluation of the Profitability of Listed Companies

Abstract: Profitability is one of the most important indexes in reflecting the performance of listed companies and it has coherences with investors, creditors and other stakeholders. So, there is great significance to rationally evaluate the earning quality conditions of the listed companies. This article employs the description method to evaluate the quality of earnings from the persistence, cash flow and reality. It designs a structural model of indexes systems and determines the importance of each index in light of their affect to the profitability of listed companies by using AHP method. Then the model provided by the research will be applied to evaluate the profitability of five wine companies of Shanghai stock exchange in 2009.

Key words: listed companies; profitability; evaluation index; AHP method

目  录

1  引  言	1
2  盈利能力评价概述	1
2.1  盈利能力概念	1
2.2  研究意义	1
3  企业盈利能力的影响因素分析	1
4  企业盈利能力评价指标体系的构建研究	1
4.1  企业盈利能力评价指标确立的一般原则	1
4.2  企业盈利能力评价指标的选择	2
4.3  企业盈利能力评价方法	2
4.3.1  构建上市公司盈利能力综合评价层次结构	2
4.3.2  构造两两比较判断矩阵	2
4.3.3  计算单一准则下元素相对排序的权重W以及一致性检验	2
4.3.4  计算层次总排序及其一致性检验	2
4.3.5  构建上市公司盈利能力综合评价的基本模型	2
5  实证分析	2
参考文献	3


1  引  言
上市公司是向社会公开发行股票而筹资成立的特殊企业,它在金融市场筹集资金,必须公开披露会计信息资料,使投资者能够详细地了解企业的财务状况、盈利能力、偿债能力、发展前景等[1]。股票投资者为了获取预计的利润、增加收入或……

2  盈利能力评价概述
2.1  盈利能力概念
企业的盈利能力,是指企业在一定时期内利用各种可获取的经济资源赚取利润的能力,也称为企业的资金或资本增值能力……
2.2  研究意义 
盈利能力是企业赖以生存的首要标志,自然而然成为经营者、投资者、债权人和其他利益相关者共同关心的焦点。对于上市公司而言……

3  企业盈利能力的影响因素分析
要考察企业的经营绩效和盈利状况,应该首先了解决定盈利水平的各种因素。对于企业来说,盈利能力的变动受多方面因素的影响,大体上可以分为内部因素和外部因素两种……

4  企业盈利能力评价指标体系的构建研究
4.1  企业盈利能力评价指标确立的一般原则
整个综合评价指标体系模式从元素到构成整体都必须合理、科学,能有效反映企业的盈利状况……
4.2  企业盈利能力评价指标的选择
4.3  企业盈利能力评价方法
由于影响企业盈利能力的因素非常多,不同的指标所反映的侧重点也各不相同,所以我们一般采用综合评价法分析企业的盈利能力……
4.3.1  构建上市公司盈利能力综合评价层次结构
企业的当前盈利性显然对于企业的综合盈利能力而言是极其重要的,而企业因盈利而实际获得的现金则为企业提供了真正的盈利保障……
4.3.2  构造两两比较判断矩阵
根据所建立的递阶层次结构体系,以上一层元素为准则,将下一层受其支配的各元素按其对上一层准则的重要程度进行两两比较……
4.3.3  计算单一准则下元素相对排序的权重W以及一致性检验
求解权值的方法很多,本文采用方根法,其计算步骤是……
4.3.4  计算层次总排序及其一致性检验
计算……
4.3.5  构建上市公司盈利能力综合评价的基本模型
1.评价指标数据的无量纲化
由于不同的指标可能数值集中区域会有所不同,为了消除由此产生的……
2.上市公司盈利能力综合评价模型的建立
上市公司的盈利能力综合评价值……

5  实证分析
现从上海证券交易所网站选取5家饮酒类上市公司,采用本文设计的上市公司盈利能力综合评价的基本模型对他们的盈利能力进行分析……


参考文献
[1] 蒋有绪,郭泉水,马娟,等.中国森林群落分类及其群落学特征[M].北京:科学出版社,1998.
[2] 阿内甘 R, 布迪尼 J.物理学教程:电学1[M].江之永,译.北京:高等教育出版社,1986.
[3] PIGGOT T M. The cataloguer’s way through AACR2: form document receipt to document retrieval[M]. London: The Library Association, 1990.
[4] 张志祥.间断动力系统的随机扰动及其在守恒律方程中的应用[D].北京: 北京大学数学学院,1998.
[5] 辛希孟.信息技术与信息服务国际研讨会论文集:A集[C].北京:科学出版社,1994.
[6] 白书农.植物开花研究[M]//李承森.植物科学进展.北京:高等教育出版社,1998:146-163.
[7] WEINSTEIN L, SWERTZ M N. Pathogenic properties of invading microor ganisma[M]//:SODEMAN W A, Jr.,SODEMAN W A. Pathologic physiology: mechanisms of disease. Philadephia: Saunder, 1974:745-772.
[8] 中国图书馆学会.图书馆学通讯[J].1957(1)-1990(4).北京:北京图书馆,1957-1990.
[9] 李晓东,张庆红,叶瑾琳.气候学研究的若干理论问题[J].北京大学学报:自然科学版,1999,35(1):101-106.
[10] 丁文祥.数字革命与竞争国际化[N].中国青年报,2000-11-20(15).
[11] 姜锡洲.一种温热外敷药制备方案:中国,88105607.3[P].1989-07-26.
[14] PACS-L:the publicaccess computer systems forum [EB/OL]. Houston, Tex:University of Houston Libraries,1989[1995-05-17].http://infor.lib.uh.edu/ pacsl.html.
\end{document}