
\chapter{插图} % 一级标题

图片通常在 $figure$ 环境中使用 $includegraphics$ 插入,如图~\ref{fig:example} 的源代码。
建议矢量图片使用 PDF 格式,比如数据可视化的绘图;
照片应使用 JPG 格式;
其他的栅格图应使用无损的 PNG 格式。
注意,LaTeX 不支持 TIFF 格式;EPS 格式已经过时。

\begin{figure}[htbp]
	\centering
	\includegraphics[width=0.5\linewidth]{example-image-a.pdf}
	%	 国外的期刊习惯将图表的标题和说明文字写成一段,需要改写为标题只含图表的名称,其他说明文字以注释方式写在图表下方,或者写在正文中。
	%	 这里将说明文字作为注释添加
	\caption{示例图片标题} % 图片标题
	\label{fig:example} % 图片标签
\end{figure}

若图或表中有附注,采用英文小写字母顺序编号,附注写在图或表的下方。

如果一个图由两个或两个以上分图组成时,各分图分别以 (a)、(b)、(c)...... 作为图序,并须有分图题。
推荐使用 $subcaption$ 宏包来处理, 比如图~\ref{fig:subfig-a} 和图~\ref{fig:subfig-b}。

\begin{figure}[htbp]
	\centering
	\begin{minipage}{0.44\linewidth}
		\centering
		\includegraphics[width=0.9\linewidth]{example-image-a.pdf}
		\caption{分图 A}
		\label{fig:subfig-a}
	\end{minipage}
	\centering
	\begin{minipage}{0.44\linewidth}
		\centering
		\includegraphics[width=0.9\linewidth]{example-image-b.pdf}
		\caption{分图 B}
		\label{fig:subfig-b}
	\end{minipage}
	\caption{多个分图的示例}
	\label{fig:multi-image}
\end{figure}