\documentclass[bachelor]{zufe}
\usepackage{setspace} % 用于设置行间距
%首航缩进
\usepackage{indentfirst}
%各种下划线和url
\usepackage{hyperref}
\usepackage{ulem}
% 图表设置
\usepackage{graphicx}
\usepackage{caption}
\usepackage{appendix}
\usepackage{lipsum}
%数学相关包
\usepackage{algorithm}
\usepackage[noend]{algpseudocode} % 使用 algpseudocode 包提供的接口
% 自定义跨页算法环境
\makeatletter
\newenvironment{breakablealgorithm}
{% \begin{breakablealgorithm}
		\begin{center}
			\refstepcounter{algorithm}% New algorithm
			\hrule height.8pt depth0pt \kern2pt% \@fs@pre for \@fs@ruled
			\renewcommand{\caption}[2][\relax]{% Make a new \caption
				{\raggedright\textbf{\textbf{Algorithm}~\thealgorithm} ##2\par}%
				\ifx\relax##1\relax % #1 is \relax
				\addcontentsline{loa}{algorithm}{\protect\numberline{\thealgorithm}##2}%
				\else % #1 is not \relax
				\addcontentsline{loa}{algorithm}{\protect\numberline{\thealgorithm}##1}%
				\fi
				\kern2pt\hrule\kern2pt
			}
		}{% \end{breakablealgorithm}
		\kern2pt\hrule\relax %\@fs@post% for \@fs@ruled
	\end{center}
}
\makeatother
%====================Tiks ====================
\usepackage{tikz}
\usetikzlibrary{shapes.geometric, arrows.meta, positioning}
\tikzstyle{process} = [rectangle, minimum width=3cm, minimum height=1cm, text centered, draw=black, fill=blue!10]
\tikzstyle{arrow} = [thick, ->, >=stealth]
%====================Tiks ====================

\usepackage{amsmath} % 支持数学符号
\usepackage{amssymb}
\usepackage{algorithmicx}
%处理数学公式中和黑斜体
\usepackage{latexsym,bm}
%定义书写公式环境
\usepackage{cases}
%一些特殊符号例如:※√
\usepackage{pifont}

% 数学命令
\makeatletter
\newcommand\dif{%  % 微分符号
	\mathop{}\!%
	\ifthu@math@style@TeX
	d%
	\else
	\mathrm{d}%
	\fi
}
\makeatother
% 表格中支持跨行
\usepackage{multirow}

% 固定宽度的表格。
\usepackage{tabularx}

% 跨页表格
\usepackage{longtable}
%给字母加粗
\usepackage[T1]{fontenc}

\usepackage{amsthm}
% 量和单位
\usepackage{siunitx}

\newcommand{\formatmain}{
	\setlength{\parskip}{0em}
	\renewcommand{\baselinestretch}{1.53}
	\interfootnotelinepenalty=10000}
%三线表
\usepackage{booktabs}

%页眉页脚
\usepackage{fancyhdr}
%三线表
\usepackage{threeparttable}
% 中英文摘要和关键字
%这里标题的设置是黑体 三号 居中
%新罗马字体
\newcommand{\romann}{\CJKfontspec{TIMES .TTF}\selectfont}
\newtheorem{theorem}{定理}

%重命名lst名字
\renewcommand{\lstlistingname}{代码}
%参考文献生成规则
\usepackage[
backend=biber,
style=gb7714-2015,
gbalign=gb7714-2015,
gbnamefmt=lowercase,
gbpub=false,
doi=false,
url=false,
eprint=false,
isbn=false,
]{biblatex}

\hypersetup{
	colorlinks=true,
	linkcolor=black,
	citecolor=black,
	urlcolor=black
}


\addbibresource{Reference.bib}

% % 使用中文编译 + XeLatex
% \usepackage[fontset=ubuntu]{ctex}

% 基本信息--------------------------------------------------------------------------------
% 在这里填写你的论文中文题目
\newcommand{\thesisTitle}{海岛环境对武学宗师成长的影响机理}
% 在这里填写你的论文英文题目
\newcommand{\thesisTitleEN}{Influence mechanism of island environment on the growth of martial arts masters}

% 若有副标题,则运行下一行的代码,若无副标题,则将下一行注释掉(在\haveSub{}最前面添加 % 号)
\haveSub{}

% 在这里填写你的论文中文副标题(没有只需注释掉\haveSub{}即可)
\newcommand{\thesisSubTitle}{基于桃花岛武学流派的研究}
% 在这里填写你的论文英文副标题(没有只需注释掉\haveSub{}即可)
\newcommand{\thesisSubTitleEN}{Research Based on Taohua island martial arts school}

% 在这里填写你的相关信息
\newcommand{\deptName}{信息管理与人工智能学院}
\newcommand{\majorName}{人工智能}
%如果是生僻字例如:垚,则需要在使用\MyFont{}将生僻字括起来,这里用的是宋体,详细可见data文件夹下的zufesetup的第五行
\newcommand{\yourName}{\MyFont{王垚垚}}
\newcommand{\yourStudentID}{210110900xxx}
\newcommand{\mentorName}{吴XX}
\newcommand{\className}{人工智能1班}
\newcommand{\Today}{2024年4月}
% 基本信息--------------------------------------------------------------------------------

% 文档开始
\begin{document}
	
	% 封面,没有特殊情况不需要修改
	% 自定义下划线命令
\newcommand\dunderline[3][-1pt]{{%
		\setbox0=\hbox{#3}%
		\ooalign{\copy0\cr\rule[\dimexpr#1-#2\relax]{\wd0}{#2}}}}

% 生成封面
\begin{titlepage}
	\makeatletter
	\centering

	\begin{center}
%  		\zihao{1}\textbf{\ziju{0.12}\songti{本\hspace{5mm}科\hspace{5mm}生\hspace{5mm}毕\hspace{5mm}业\hspace{5mm}论\hspace{5mm}文(设计)}}
		\zihao{1}\textbf{\ziju{0.12}{MASTER THESIS}}
	
		\vspace{24mm}
		
		\xiaoyi\textbf{A Report on the Translation of \textsl{\thesisname}}
		
		\vspace{18mm}
		\sanhao{Master of Translation and Interpreting}
		
		\vspace{6mm}
		\sanhao{School of Foreign Languages}

		\vspace{6mm}
		\sanhao{Zhejiang University of Finance and Economics}
		
		\vspace{12mm}
		\sanhao{Submitted by \textbf{\studentname}}
		
		\vspace{6mm}
		\sanhao{E-mail: \url{\studenteml}}
		
		\vspace{9mm}
		\textbf{\sanhao Supervisor: {\advisor}}
		
		\vspace{30mm}
		\centering
		\textbf{\sanhao {\thesisdate}}
	\end{center}
\end{titlepage}
	
	% 前置页面定义
	\frontmatter
	% 原创性声明,没有特殊情况不需要修改
	\input{misc/1_originality}
	
	% 摘要(中英文):根据自身论文,修改摘要的Tex文件
	
% 下面是中文摘要 3-19行不用管
\setstretch{2}
\vspace*{2em}

\begin{center}
	\zihao{3}\textbf{ABSTRACT}
	\vspace{2mm}
	\ifx\subtitle\undefined\else
	%如果有副标题
	\begin{spacing}{1.2}
		\sanhao\selectfont{\textmd{\kaishu{-----}\subtitle}}
	\end{spacing}
	\fi
\end{center}
%题目后空一行
\vspace{1em}
\setstretch{1.5}

\setlength{\parskip}{0em}

% 英文摘要正文从这里开始--------------------------------------------------
As economic globalization continues to advance in depth, the translation of academic literature on economics has increasingly demonstrated its significance. This report selects and translates the 2015 Nobel Prize laureate’s academic literature into Chinese. Based on the research of the selected text and its Chinese translation by combining the case analysis with the comparative analysis, the report attempts to focus on what translation strategies are more effective in terms of realizing the ideational, interpersonal, and textual functions of the source text. The selected materials of this translation practice report do not have any existing translation versions for reference and the present author uses computer-aided technology to conduct a data analysis of the full original text before translation so as to learn about the features of the original.
\par
This report is guided by the three meta-functions of Halliday’s systemic functional linguistics. Through the analysis, it is concluded that annotation of terminologies, division of long sentences, and conversion of passive voice can achieve the ideational functions of the translation; addition of personal pronouns, literal translation of modal verbs and substitution of mood adjuncts can realize the interpersonal function; omission of connectives and conjunctive adjuncts, reproduction of demonstrative pronouns and adverbs, and reorganization of sentence order and paragraphs can achieve the textual function. The practicability of these strategies has been confirmed by concrete examples in this report. For the translation of academic literature on economics, comprehension and expression are pivotal. The translator should be familiar with the characteristics and style of the text, proficient in translation skills with the help of computer-aided tools, and flexible in dealing with rendering difficulties. This report is also intended to give some implications to both the translation practice of and theoretical application to the similar texts in the future.
\par

\vspace{1em}
\textbf{\heiti\xiaosi{Key words:}}
%这里写你的关键词------------------------------------------------------
academic literature on economics; three meta-functions of language; translation strategies

\newpage
\setstretch{2}
\vspace*{2em}
%如果没有英文标题,跳过
%下面是英文摘要 37-52行不用管
\begin{spacing}{0.95}
	\centering
	\sanhao\textbf{\heiti{摘要}}
	\vspace{2mm}
	\ifx\subtitle\undefined\else
	%如果有副标题
	\begin{spacing}{1.2}
		\sihao\selectfont{\textmd{\kaishu{-----}\rmfamily\textbf{\ensubtitle}}}
	\end{spacing}
	\fi
\end{spacing}
\vspace{1em}
\setstretch{1.5}
\setlength{\parskip}{0em}

% 中文摘要正文从这里开始--------------------------------------------------
随着经济全球化不断纵深推进,经济学类学术文献的翻译愈发显示出其重要性。本报告选择2015年诺贝尔经济学奖获得者的学术讲演稿进行汉译,基于对所选文本及其汉译文的研究,结合个案分析法与对比分析法,试图重点考察经济学类学术文献的翻译过程中采用何种策略可以在译文中更为有效地实现概念功能、人际功能、语篇功能等三大语言纯理功能。本翻译实践报告的所选材料无任何现有译文可供参考,笔者为了解原文特点,在译前借助计算机辅助技术对全文进行了数据分析。
\par
本报告以韩礼德系统功能语言学理论的三大语言纯理功能为指导,通过分析,结论如下:对术语加注、拆分长句和转换被动语态能够实现译文的概念功能;增译人称代词、直译情态动词和替换语气附加语可以实现译文的人际功能;省却连词和连接附加语、再现指示代词和指示副词,以及重组句序和段落能够实现译文的语篇功能,这些策略的实用性已由本报告的翻译实例得到印证。对于经济学类学术文献的翻译,理解能力与表达方式至关重要,译者当熟悉文体特征,精于翻译技巧并辅以机辅工具,灵活应对翻译难点。本报告对今后类似文本的翻译实践及理论应用也具有一定的启示意义。
\par

\vspace{1em}
\textbf{\rmfamily\xiaosi{关键词:}}
%这里写你的英文关键词----------------------------------------------------
经济学类学术文献;三大语言纯理功能;翻译策略



	%是否需要换页自己判断
	\newpage
	
	% 目录,自动生成,没有特殊情况不需要修改
	% 论文目录
% 没有特殊需要不用修改

% 调整目录行间距
\renewcommand{\baselinestretch}{1.35}
% 目录
\mainmatter
\addtocontents{toc}{\protect\thispagestyle{empty}}
\pagenumbering{gobble}
\tableofcontents
\clearpage
\pagenumbering{arabic}

	% 正文开始
	\mainmatter
	% 正文 22 磅的行距
	\setlength{\parskip}{0em}
	\renewcommand{\baselinestretch}{1.53}
	
	% 修复脚注出现跨页的问题
	\interfootnotelinepenalty=10000
	
	% 引言,根据自身论文,修改引言的Tex文件
	\input{misc/3_introduction}
	\newpage
	
	% 章节模板,成文后将其注释掉即可
	% !TeX root = ../zufe_report.tex
\setlength{\baselineskip}{20pt}
% 样例章节

% 章和引用示例-------------------------------------------------------------------------------------
\chapter{一级题目}

\section{二级题目}

正文······\cite{GB/T16159—1996} % 引用示例

\subsection{三级题目}

正文······\cite{Sobieski}
% 章和引用示例-------------------------------------------------------------------------------------


% 表示例----------------------------------------------------------------------------------------
\section{\textcolor{blue}{\underline{\underline{表-示例}}}}

\textcolor{blue}{{自动生成LaTeX表工具: \url{https://www.tablesgenerator.com/}}}

\begin{table}[htbp]
  \linespread{1.5}
  \zihao{5}
  \songti
  \centering
  \caption{物流的概念和范围}\label{物流的概念和范围}
  \begin{tabular}{|l|l|}
  \hline
  \multicolumn{1}{|c|}{本 质} & \multicolumn{1}{c|}{过  程}  \\ \hline
  途径或方法                     & 规划、实施、控制                   \\ \hline
  目标                        & 效率、成本效益                    \\ \hline
  活动或作业                     & 流动与储存                      \\ \hline
  处理对象                      & 原材料、在制品、产成品、相关信息           \\ \hline
  范围                        & 从原点(供应商)到终点(最终顾客)          \\ \hline
  目的或目标                     & 适应顾客的需求(产品、功能、数量、质量、时间、价格) \\ \hline
  \end{tabular}
\end{table}

 \begin{table}[htbp]
   \linespread{1.5}
   \zihao{5}
   \songti
   \centering
   \caption{统计表}\label{统计表}
   \begin{tabular}{|l|l|l|l|l|}
   \hline
   产品  & 产量    & 销量    & 产值   & 比重    \\ \hline
   手机  & 11000 & 10000 & 500  & 50\%  \\ \hline
   电视机 & 5500  & 5000  & 220  & 22\%  \\ \hline
   计算机 & 1100  & 1000  & 280  & 28\%  \\ \hline
   合计  & 17600 & 16000 & 1000 & 100\% \\ \hline
   \end{tabular}
 \end{table}

\begin{table}[htbp]
  \linespread{1.5}
  \zihao{5}
  \songti
  \centering
  \caption{统计表}\label{统计表}
  % Please add the following required packages to your document preamble:
  % \usepackage{multirow}
  \begin{tabular}{|l|l|l|l|l|}
  \hline
  年度                    & 产品  & 产量    & 销量    & 产值  \\ \hline
  \multirow{2}{*}{2004} & 手机  & 11000 & 10000 & 500 \\ \cline{2-5} 
                      & 计算机 & 1100  & 1000  & 280 \\ \hline
  \multirow{2}{*}{2005} & 手机  & 16000 & 13000 & 550 \\ \cline{2-5} 
                      & 计算机 & 2100  & 1500  & 320 \\ \hline
  \end{tabular}
\end{table}
% 表示例----------------------------------------------------------------------------------------

\section{\textcolor{blue}{\underline{\underline{表格-示例}}}}


表应具有自明性。为使表格简洁易读,尽可能采用三线表,如表~\ref{tab:three-line}。
三条线可以使用 booktabs 宏包提供的命令生成。

\begin{table}[htbp]
	\centering
	\caption{三线表示例}
	\begin{tabular}{ll}
		\toprule
		文件名          & 描述                         \\
		\midrule
		zufereport.dtx   & 模板的源文件,包括文档和注释 \\
		zufereport.cls   & 模板文件                     \\
		zufereport-*.bst & BibTeX 参考文献表样式文件    \\
		\bottomrule
	\end{tabular}
	\label{tab:three-line}
\end{table}

表格如果有附注,尤其是需要在表格中进行标注时,可以使用 threeparttable 宏包。
研究生要求使用英文小写字母 a、b、c……顺序编号,本科生使用圈码 ①、②、③……编号。

\begin{table}[htbp]
	\centering
	\begin{threeparttable}[c]
		\caption{带附注的表格示例}
		\label{tab:three-part-table}
		\begin{tabular}{ll}
			\toprule
			文件名                 & 描述                         \\
			\midrule
			zufereport.dtx\tnote{a} & 模板的源文件,包括文档和注释 \\
			zufereport.cls\tnote{b} & 模板文件                     \\
			zufereport-*.bst        & BibTeX 参考文献表样式文件    \\
			\bottomrule
		\end{tabular}
		\begin{tablenotes}
			\item [a] 可以通过 xelatex 编译生成模板的使用说明文档;
			使用 xetex 编译 \file{zufereport.ins} 时则会从 \file{.dtx} 中去除掉文档和注释,得到精简的 \file{.cls} 文件。
			\item [b] 更新模板时,一定要记得编译生成 \file{.cls} 文件,否则编译论文时载入的依然是旧版的模板。
		\end{tablenotes}
	\end{threeparttable}
\end{table}

如某个表需要转页接排,可以使用 longtable 宏包,需要在随后的各页上重复表的编号。
编号后跟表题(可省略)和“(续)”,置于表上方。续表均应重复表头。

\singlespacing
\begin{longtable}{cccc}
	\caption{跨页长表格的表题}
	\label{tab:longtable} \\
	\toprule
	表头 1 & 表头 2 & 表头 3 & 表头 4 \\
	\midrule
	\endfirsthead
	\caption*{续表~\thetable\quad 跨页长表格的表题} \\
	\toprule
	表头 1 & 表头 2 & 表头 3 & 表头 4 \\
	\midrule
	\endhead
	\bottomrule
	\endfoot
	Row 1  & & & \\
	Row 2  & & & \\
	Row 3  & & & \\
	Row 4  & & & \\
	Row 5  & & & \\
	Row 6  & & & \\
	Row 7  & & & \\
	Row 8  & & & \\
	Row 9  & & & \\
	Row 10 & & & \\
\end{longtable}

% 伪代码示例------------------------------------------------------------------------------------

\section{\textcolor{blue}{\underline{\underline{伪代码-示例}}}}

\textcolor{blue}{修改 algorithmic 之间的代码就可以实现论文伪代码,已经考虑了伪代码跨页问题}
\begin{breakablealgorithm}
        \caption{Calculate $y = x^n$}
        \begin{algorithmic}[1] %每行显示行号
            \Require $n \geq 0 \vee x \neq 0$ 
            \Ensure  $y = x^n$
            \State $y \gets 1$
            \If{$n < 0$}
            \State $ X \gets 1 / x$
            \State $N \gets -n$
            \Else
            \State $X \gets x$
            \State $N \gets n$
            \EndIf
            \While{ $N \neq 0$ }
            \If{ $N$ is even }
            \State $X \gets x \times x$
            \State $N \gets N / 2$
            \Else[$N$ is odd]
            \State $y \gets y \times X$
            \State $N \gets N - 1$
            \EndIf
            \EndWhile
     \end{algorithmic}
    \end{breakablealgorithm}
% 伪代码示例------------------------------------------------------------------------------------

% 代码块示例------------------------------------------------------------------------------------
\section{\textcolor{blue}{\underline{\underline{代码块-示例}}}}
\textcolor{blue}{只写了 C++, Python, Java 三种语言的格式}

% C++ 代码引入 + 文件引入
\lstinputlisting[
style = C++,
caption = {test.cpp},
label = {test.cpp}
]{./papperCode/test.cpp}

% Java 代码引入 + 行内引入
\begin{lstlisting}[
style = Java,
caption = {test.jar},
label = {test.jar}
]
public class HelloWorld {
    public static void main(String[] args){
        System.out.println("Hello World!");
    }
}
\end{lstlisting}

% Python 代码引入 + 文件引入
\lstinputlisting[
style = Python,
caption = {test.py},
label = {test.py}
]{./papperCode/test.py}
% 代码块示例------------------------------------------------------------------------------------
	\newpage
	
	% 在此插入章节
	% 第一章
	\input{chapters/chapter_1}
	\newpage
	
	% 第二章,第三章······
	% !TeX root = ../zufe_report.tex
\setlength{\baselineskip}{20pt}
\section{表格示例}
\subsection{表格}

表应具有自明性。为使表格简洁易读,尽可能采用三线表,如表~\ref{tab:three-line}。
三条线可以使用 booktabs 宏包提供的命令生成。

\begin{table}[htbp]
	\centering
	\caption{三线表示例}
	\begin{tabular}{ll}
		\toprule
		文件名          & 描述                         \\
		\midrule
		zufereport.dtx   & 模板的源文件,包括文档和注释 \\
		zufereport.cls   & 模板文件                     \\
		zufereport-*.bst & BibTeX 参考文献表样式文件    \\
		\bottomrule
	\end{tabular}
	\label{tab:three-line}
\end{table}

表格如果有附注,尤其是需要在表格中进行标注时,可以使用 threeparttable 宏包。
研究生要求使用英文小写字母 a、b、c……顺序编号,本科生使用圈码 ①、②、③……编号。

\begin{table}[htbp]
	\centering
	\begin{threeparttable}[c]
		\caption{带附注的表格示例}
		\label{tab:three-part-table}
		\begin{tabular}{ll}
			\toprule
			文件名                 & 描述                         \\
			\midrule
			zufereport.dtx\tnote{a} & 模板的源文件,包括文档和注释 \\
			zufereport.cls\tnote{b} & 模板文件                     \\
			zufereport-*.bst        & BibTeX 参考文献表样式文件    \\
			\bottomrule
		\end{tabular}
		\begin{tablenotes}
			\item [a] 可以通过 xelatex 编译生成模板的使用说明文档;
			使用 xetex 编译 \file{zufereport.ins} 时则会从 \file{.dtx} 中去除掉文档和注释,得到精简的 \file{.cls} 文件。
			\item [b] 更新模板时,一定要记得编译生成 \file{.cls} 文件,否则编译论文时载入的依然是旧版的模板。
		\end{tablenotes}
	\end{threeparttable}
\end{table}

如某个表需要转页接排,可以使用 \usepackage{longtable} 宏包,需要在随后的各页上重复表的编号。
编号后跟表题(可省略)和“(续)”,置于表上方。续表均应重复表头。

\singlespacing
\begin{longtable}{cccc}
	\caption{跨页长表格的表题}
	\label{tab:longtable} \\
	\toprule
	表头 1 & 表头 2 & 表头 3 & 表头 4 \\
	\midrule
	\endfirsthead
	\caption*{续表~\thetable\quad 跨页长表格的表题} \\
	\toprule
	表头 1 & 表头 2 & 表头 3 & 表头 4 \\
	\midrule
	\endhead
	\bottomrule
	\endfoot
	Row 1  & & & \\
	Row 2  & & & \\
	Row 3  & & & \\
	Row 4  & & & \\
	Row 5  & & & \\
	Row 6  & & & \\
	Row 7  & & & \\
	Row 8  & & & \\
	Row 9  & & & \\
	Row 10 & & & \\
\end{longtable}


\subsection{算法}

算法环境可以使用 algorithms 或者 algorithm2e 宏包。

\renewcommand{\algorithmicrequire}{\textbf{输入:}\unskip}
\renewcommand{\algorithmicensure}{\textbf{输出:}\unskip}

\begin{algorithm}
	\caption{Calculate $y = x^n$}
	\label{alg1}
	\small
	\begin{algorithmic}
		\REQUIRE $n \geq 0$
		\ENSURE $y = x^n$
		
		\STATE $y \leftarrow 1$
		\STATE $X \leftarrow x$
		\STATE $N \leftarrow n$
		
		\WHILE{$N \neq 0$}
		\IF{$N$ is even}
		\STATE $X \leftarrow X \times X$
		\STATE $N \leftarrow N / 2$
		\ELSE[$N$ is odd]
		\STATE $y \leftarrow y \times X$
		\STATE $N \leftarrow N - 1$
		\ENDIF
		\ENDWHILE
	\end{algorithmic}
\end{algorithm}

	\newpage
	
	% 结论:根据自身论文,修改结论的Tex文件
	\input{misc/4_conclusion}
	\newpage
	
	% 参考文献,无特殊要求不需要修改
	% 添加参考文献请使用 BibTex 格式,添加至reference.bib 中,并在正文中使用 \cite{xxx}
	% 成文后将\input{misc/5_reference}注释掉即可,这个只是告诉你参考文献的类别以及规范
	\input{misc/5_reference}
	\input{misc/5_simple_reference}
	\newpage
	
	% 附录:根据自身论文,修改附录的Tex文件(补充论文,不是必须)
	\input{misc/6_appendix}
	\newpage

	% 致谢:根据自身要求,修改致谢的Tex文件
	\input{misc/7_acknowledgement}
\end{document}
