\RequirePackage{CJKnumb}
%生僻字
\usepackage{ctex}
\setCJKfamilyfont{myfont}{simsun.ttc}
\newcommand{\MyFont}{\CJKfamily{myfont}}
%参考文献
\usepackage[backend=biber,style=numeric-comp,sorting=nty]{biblatex}
%参考文献出现在上标
\newcommand{\mycite}[1]{\textsuperscript{\cite{#1}}}

%超链接
\usepackage[colorlinks=true,linkcolor=black,urlcolor=black,citecolor=black]{hyperref}
\newcommand{\itemcmd}[1]{#1}
\usepackage{caption}

%数学相关包
\RequirePackage{amsmath}
\usepackage{algorithm}  
\usepackage{algorithmicx}  
\usepackage{algpseudocode}
\RequirePackage{amsfonts}
\RequirePackage{amssymb}
\RequirePackage{bm}
\usepackage{upgreek}
% 数学命令
\makeatletter
\newcommand\dif{%  % 微分符号
	\mathop{}\!%
	\ifthu@math@style@TeX
	d%
	\else
	\mathrm{d}%
	\fi
}
\makeatother
% 表格中支持跨行
\usepackage{multirow}

% 固定宽度的表格。
\usepackage{tabularx}

% 跨页表格
\usepackage{longtable}
%给字母加粗
\usepackage[T1]{fontenc}

\usepackage{amsmath}
\usepackage{amsthm}
% 量和单位
\usepackage{siunitx}


%三线表
\usepackage{booktabs}

%调用子图
\usepackage{subfigure}
%页眉页脚
\usepackage{fancyhdr}
%三线表
\usepackage{threeparttable}
% 中英文摘要和关键字
%这里标题的设置是黑体 三号 居中
%新罗马字体
\newcommand{\romann}{\CJKfontspec{TIMES.TTF}\selectfont}
%中英文摘要
\newcommand{\enabstractname}{}
\newcommand{\cnabstractname}{}
\newenvironment{enabstract}{%
	\par\small
	\noindent\mbox{}\hfill{\bfseries\enabstractname}\hfill\mbox{}\par
	\vskip 2.5ex}{\par\vskip2.5ex}
\newenvironment{cnabstract}{%
	\par\small
	\noindent\mbox{}\hfill{\bfseries\cnabstractname}\hfill\mbox{}\par
	\vskip 2.5ex}{\par\vskip2.5ex}
{\tiny {\tiny }}
%%%%%%%%%%%  设置字体大小 ,拼音代表字号%%%%%%%%%%%%%
\newcommand{\xiaochuhao}{\fontsize{36pt}{\baselineskip}\selectfont}
\newcommand{\xiaoerhao}{\fontsize{18pt}{\baselineskip}\selectfont}
\newcommand{\xiaosihao}{\fontsize{12pt}{\baselineskip}\selectfont}
\newcommand{\xiaowuhao}{\fontsize{9pt}{\baselineskip}\selectfont}
\fancyhead{}

\newtheorem{theorem}{定理}