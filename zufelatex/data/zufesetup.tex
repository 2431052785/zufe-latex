\usepackage{setspace} % 用于设置行间距
%首航缩进
\usepackage{indentfirst}
%各种下划线和url
\usepackage{hyperref}
\usepackage{ulem}
% 图表设置
\usepackage{graphicx}
\usepackage{caption}
\usepackage{appendix}
\usepackage{lipsum}
%数学相关包
\usepackage{algorithm}
\usepackage[noend]{algpseudocode} % 使用 algpseudocode 包提供的接口
% 自定义跨页算法环境
\makeatletter
\newenvironment{breakablealgorithm}
{% \begin{breakablealgorithm}
		\begin{center}
			\refstepcounter{algorithm}% New algorithm
			\hrule height.8pt depth0pt \kern2pt% \@fs@pre for \@fs@ruled
			\renewcommand{\caption}[2][\relax]{% Make a new \caption
				{\raggedright\textbf{\textbf{Algorithm}~\thealgorithm} ##2\par}%
				\ifx\relax##1\relax % #1 is \relax
				\addcontentsline{loa}{algorithm}{\protect\numberline{\thealgorithm}##2}%
				\else % #1 is not \relax
				\addcontentsline{loa}{algorithm}{\protect\numberline{\thealgorithm}##1}%
				\fi
				\kern2pt\hrule\kern2pt
			}
		}{% \end{breakablealgorithm}
		\kern2pt\hrule\relax %\@fs@post% for \@fs@ruled
	\end{center}
}
\makeatother
%====================Tiks ====================
\usepackage{tikz}
\usetikzlibrary{shapes.geometric, arrows.meta, positioning}
\tikzstyle{process} = [rectangle, minimum width=3cm, minimum height=1cm, text centered, draw=black, fill=blue!10]
\tikzstyle{arrow} = [thick, ->, >=stealth]
%====================Tiks ====================

\usepackage{amsmath} % 支持数学符号
\usepackage{amssymb}
\usepackage{algorithmicx}
%处理数学公式中和黑斜体
\usepackage{latexsym,bm}
%定义书写公式环境
\usepackage{cases}
%一些特殊符号例如:※√
\usepackage{pifont}

% 数学命令
\makeatletter
\newcommand\dif{%  % 微分符号
	\mathop{}\!%
	\ifthu@math@style@TeX
	d%
	\else
	\mathrm{d}%
	\fi
}
\makeatother
% 表格中支持跨行
\usepackage{multirow}

% 固定宽度的表格。
\usepackage{tabularx}

% 跨页表格
\usepackage{longtable}
%给字母加粗
\usepackage[T1]{fontenc}

\usepackage{amsthm}
% 量和单位
\usepackage{siunitx}

\newcommand{\formatmain}{
	\setlength{\parskip}{0em}
	\renewcommand{\baselinestretch}{1.53}
	\interfootnotelinepenalty=10000}
%三线表
\usepackage{booktabs}

%页眉页脚
\usepackage{fancyhdr}
%三线表
\usepackage{threeparttable}
% 中英文摘要和关键字
%这里标题的设置是黑体 三号 居中
%新罗马字体
\newcommand{\romann}{\CJKfontspec{TIMES .TTF}\selectfont}
\newtheorem{theorem}{定理}

%重命名lst名字
\renewcommand{\lstlistingname}{代码}
%参考文献生成规则
\usepackage[
backend=biber,
style=gb7714-2015,
gbalign=gb7714-2015,
gbnamefmt=lowercase,
gbpub=false,
doi=false,
url=false,
eprint=false,
isbn=false,
]{biblatex}

\hypersetup{
	colorlinks=true,
	linkcolor=black,
	citecolor=black,
	urlcolor=black
}

