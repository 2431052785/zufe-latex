\chapter*{Summary}
\addcontentsline{toc}{chapter}{Summary}
%下面是总结的内容
The report chooses “Measuring and Understanding Behavior, Welfare, and Poverty”, an academic literature on economics written by the Nobel laureate Deaton in 2015, as the source text to be translated into Chinese. “Measuring and Understanding Behavior, Welfare, and Poverty” stands as a prestigious passage as well as a revealing speech concerning the issues of international poverty alleviation, consumer behavior and social welfare; it mainly explores and illustrates the outcomes of home surveys worldwide, trying to make thought-provoking explanations about the economic disparities among different countries and regions. Featured with a battery of terminologies and professional expressions, this text is also deluged with long academic sentences, passive voice, personal and demonstrative pronouns and so forth, which poses major obstacles during the translation process for a nonprofessional in the field of economics. Under the guidance of three meta-functions of systemic functional linguistics, the present author deliberately adopts and then meticulously analyzes the translation strategies to remove the lexical difficulties by way of annotation of terminologies, division of long sentences, and conversion of passive voice; to tackle the syntactic difficulties through addition of personal pronouns, literal translation of modal verbs and substitution of mood adjuncts; and to deal with the textual difficulties by means of omission of connectives and conjunctive adjuncts, reproduction of demonstrative pronouns and adverbs, and reorganization of sentence order and paragraphs, thus enabling the translated version to be faithful, logical as well as apprehensible for both professionals and nonprofessionals of economics.

The interim conclusions of this report can be summarized as follows:

Firstly, based upon the translation of academic literature on economics of this kind, the present author holds that the most elementary and essential quality for a translator is the comprehension ability and expression capability since these two competences can make a huge difference to the final version, either to the translation’s meaning, or to its wording. For this reason, the present author pays much more attention to academic or professional information in the source text as well as the characteristics and style of the source text, so as to keep away from misunderstandings or misinterpretations to the largest extent.
Secondly, ...

Thirdly, ...

Limitations, however, arise in the report. Firstly, this report is not a complete summary and discussion about the translation strategies which can be skillfully adopted in the rendering process of academic literature on economics; secondly, nine kinds of translation strategies mentioned above haven’t yet been proved whether they are universally applicable, equally effective in other kinds of relative texts including scientific articles and expository writings.

As a result, it is suggested that a more systematic and comprehensive study of translation strategies for academic literature on economics need to be made. Moreover, ...
