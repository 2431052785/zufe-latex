\chapter{Process Description}
\section{Pre-translation Preparations}
	\subsection{Translator’s Role and Qualification}
	Translation is an activity in which one language is skillfully transformed into another, and a translator is just like an assistant, an intermediary or a coordinator during the rendering process. With respect to this academic literature on economics, the present author-cum-translator shall take the responsibility to act as a bridge in order to correctly and appropriately convey both the message and the emotional propensity contained in the source text to the target audience. The most important role that the translator is ought to play is an eligible communicator equipped with translation strategies and bi-cultural awareness.
	\subsection{Related Materials, Translation and Corpus Retrieval Tools}
	The present author finds some authoritative bilingual dictionaries like The English-Chinese Dictionary (Second Edition) and The Chinese-English Dictionary (Second Edition) compiled by Lu Gusun, and collects several related economics books about specific terms contained in the literature as well as Deaton’s own books and passages in advance, among which the books are Economics and Consumer Behavior, Understanding consumption, The Great Escape: Health, Wealth, and the Origins of Inequality, and articles are “COVID-19 and Global Income Inequality,” “GDP, Wellbeing, and Health: Thoughts on the 2017 Round of the International Comparison Program,” “Trying to Understand the PPPs in ICP2011: Why are the Results so Different?” and “Creative Destruction and Subjective Wellbeing.”
	
	...
	\subsection{ Quality-control Methods}
	There are three quality-control methods the present author adopts in her translation process so that she can revise and edit her translation version. They are as follows:
	...
	
\section{During-translation}
	\subsection{Understanding the Source Text}
	Understanding the source text mainly includes the understanding of the writer’s information, the target readers’ expectation, and of course, the comprehension of the source text; among them, the most essential part of understanding has to be the last one, which involves the analysis concerning semantic and syntactic features of this literature as well as Deaton’s writing style, et al. 
	...
	
\section{ Difficulties in Translation}
	\subsection{ Knowledge about Economics}
	Economics knowledge poses an enormous obstacle when the present author translates the text, an academic literature written by the Nobel Laureate who specializes in micro-economics for decades, while the translator has very limited knowledge in this field before she gets started. During the rendering process, the present author gradually discovers that without enough economic background knowledge, the translator may feel puzzled when she encounters some specific terminologies and unique expressions from time to time, and that the most head-scratching thing is that the word itself is common and quite familiar to the present author, for instance, the real meaning of “AIDS” in this source text, however, is completely different from what the translator presumes -- a brand-new economics model. 
	...