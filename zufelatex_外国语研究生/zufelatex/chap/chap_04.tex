\chapter{Case Analysis}
After a brief introduction to the functional grammar proposed by Halliday, what is going to present is how to apply his three meta-functions theory, as a beacon, to guide the Chinese translation of “Measuring and Understanding Behavior, Welfare, and Poverty”. As is mentioned above, the ideational function, the interpersonal function and the textual function respectively vary in the form and can be embodied by different expressions; based on that, the present author meticulously chooses the following representative translation strategies applied in her own rendering so as to illustrate how these tactics, serving as the catalyst of the final translation version, make clear original meanings, contexts and inner logic of the source text, showcase the laureate’s explicit and implicit emotions, judgments or attitudes, as well as reveal the correlation and interrelationship within the whole article.
To be specific, the case analysis consists of three sections. Firstly, ideational function and translation of academic literature on economics, including annotation of terminologies, division of long sentences and conversion of passive voice. Secondly, interpersonal function and translation of academic literature on economics, including addition of personal pronouns, literal translation of modal verbs and substitution of mood adjuncts. Thirdly, textual function and translation of academic literature on economics, including omission of connectives and conjunctive adjuncts, reproduction of demonstrative pronouns and adverbs as well as reorganization of sentence order and paragraphs. 
\section{Ideational Function and Translation of Academic Literature on Economics}
As for the ideational function, translation of academic literature on economics mainly focuses on the experiential function. That is to say, the translation lays more emphasis on strategies applied in descriptions and depictions of economics-related concepts, jargons and knowledge, background information, historical development, and so forth. Besides, ideational function intends to accurately express the original meanings and deliberately shuns any ambiguity or obscurity, which can be seen from the translation strategies over long sentences and passive voice in particular.
	\subsection{Annotation of Terminologies }
	A crowd of terminologies are part of the reason why the chosen literature “Measuring and Understanding Behavior, Welfare, and Poverty” is indisputably academic and abstruse for target audience, rich in terminologies. Extra explanations or annotations in the context of translation studies can be of great help for further understanding.
	
	Example (1) ST: preference (Appendix A: 52)
	
	TT: (个人)偏好,微观经济学效用理论中的一个概念,即指消费者对某商品或商品组合的喜好程度或决策人对收益和风险的态度。消费者对商品的偏好可以基于客观指标,也可以基于因心理感受作出的主观判断。每位消费者都拥有特定的偏好,产生独有的价值判断,并在个人行为中据此对商品及其数量所带来满足程度的高低进行排序。(Appendix A: 52, 94)
	
	Take existing academic books as reference, the present author translates “preference” into “(个人)偏好” which is one of the basic hypotheses in microeconomics, referring to the degree to which a customer prefers a product or a commodity combination. Annotation of this special term can be of great help in terms of explaining the customers’ subjectivity and value judgment towards products and elucidating that each customer has his own preference. Therefore, the annotation is introducing the economic background information to target readers in order to help them have a better ideational understanding.
	...
\section{Interpersonal Function and Translation of Academic Literature on Economics}

As for the interpersonal function, translation of academic literature on economics mainly concentrates on the description and expression of the writer’s identity, position, attitude or motivation, as well as on the delineation of his inference, judgment or evaluation about related issues. Similarly, there are lots of translation methods that can be applied to prop up the interpersonal function, including those dealing with personal pronoun, modal verb, mood adjunct, and so on. The ultimate goal of the interpersonal function is to establish a sort of specific relationship with target audience and then arouse their empathy.
	\subsection{Addition of Personal Pronouns}
	In the Chinese language, personal pronouns generally are less common than in the English language. Nevertheless, translating this academic literature requires the translator to act according to specific contexts, which means to add some personal pronouns in the rendering if necessary, so as to clarify the subjects and thus cater to the need of academic accuracy.
	
	Example (13) ST: Aggregation needs to be seen, not as a nuisance, but as a hallmark of seriousness, as well as a source of hypotheses and understanding. (Appendix A: 51) 
	
	TT: 我们的确需要关注整体,不能将其当作一件麻烦事而等闲视之,相反要把整体作为科研严谨的标识,亦作为学说和理解的来源。(Appendix A: 51)
	
	The present author adds the personal pronoun “我们” when she translates the expression “aggregation needs to be seen”. In the target language, the version “我们的确需要关注整体” is a much more common and familiar expression before stating a watershed perspective, aiming to convince target readers that like most people, the writer does indeed acknowledge and share the former idea, but he will point out another statement then to explain his thoughts.
	...
	
\section{Textual Function and Translation of Academic Literature on Economics}
As for the textual function, translation of academic literature on economics lays more emphasis on the connection and coherence in the whole text as well as between sentences and paragraphs. As is mentioned in An Introduction to Functional Grammar, the systems of cohesion operate within either the grammatical zone or the lexical zone of the lexicogrammatical continuum (Halliday, 2004: 538), so there are two primary approaches to realize this function: lexical cohesion and grammatical cohesion. Specifically, translation strategies (omission, reproduction and reorganization) applied in the light of connectives and conjunctive adjunct, demonstrative pronouns and adverbs as well as sentence order and paragraphs are fairly conducive to translating the rigorous thinking and logic of this literature.
	\subsection{Omission of Connectives and Conjunctive Adjuncts}
	The conjunctive adjuncts cover roughly the same semantic space as the conjunctions; but whereas conjunctions set up a grammatical (systemic-structural) relationship with another clause, which may be either preceding or following, the relationship established by conjunctive adjuncts, while semantically cohesive, is not a structural one (Halliday, 2004: 83). Omission of these words in the Chinese version can undoubtedly achieve communication and also pander to the taste of the Chinese readers.
	
	Example (25) ST: The work cited by the Prize Committee spans many years, covers areas of economics that are not always grouped together, and involves many different collaborators. (Appendix A: 50)
	
	TT: 诺贝尔奖委员会所引用的这项研究,旷日持久,涵盖了经济学中常常分而治之的领域,期间幸得多方协助才得以完成。(Appendix A: 50)
	In the above example, the present author deliberately omits “and” in “and involves many different collaborators”. In Chinese, it is common and idiomatic that connectives are not required between short sentences, from which no ambiguity is supposed to arise. From the angle of textual function, the omission of “and” here makes the translation much more authentic in the target language and more fluent and coherent in the situation as well.
	Example (26) ST: Parenthetically, Stone seemed to be unaware of what is now (somewhat ironically) called the Stone-Geary utility function (Cobb Douglas with an affine shift of origin). (Appendix A: 72)
	
	TT: 顺便提一句,有点讽刺的是,斯通当时似乎并未意识到这即是所谓的“斯通-格瑞效用函数(the Stone-Geary utility function)”(也就是原点具有仿射位移的柯布-道格拉斯函数)。 (Appendix A: 72)
	
	The conjunctive adjunct “now” is also omitted in this example when the present author translates “what is now ... called the Stone-Geary utility function” into Chinese “这即是所谓的‘斯通-格瑞效用函数’”. Through omission of the “redundant” adjunct, the translator renders the original meaning of the source text properly; it is superseded by the internal logic of Chinese language to coherently comb the history of this prestigious function of economics.
	...