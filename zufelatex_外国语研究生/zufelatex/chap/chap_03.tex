\chapter{Guiding Theory}
\section{Overview of Functional Grammar}
During the period of 1960s to 1970s, systemic functional linguistics began to sprout up and gradually blazed its new trail in language studies. Systemic functional linguistics proposed by Halliday is a kind of general linguistics as well as an appliable linguistics that he defines as a comprehensive, theoretically powerful language model, intending to address language-related problems in theory and practice. Another name “Neo-Firthian linguistics” reveals the fact that systemic functional linguistics is directly stemmed from the London school with representatives like J. R. Firth, and later on, is collectively enriched by thoughts of K. Buhler in the Prague school, sociologist B. Bernstein, anthropologist B. Malinowski as well as some other linguistic theorists, such as S. Lamb, C. J. Fillmore and K. C. Pike. 
...
\section{Three Meta-functions}
Based upon massive records and analysis of early language development in children, it is believed by Halliday that language development in children is “the mastery of linguistic functions”, and “learning a language is learning how to mean” (Hu Zhuanglin, 2006: 311). Therefore, he puts forward seven functions in children’s model of language: the instrumental function, the regulatory function, the interactional function, the personal function, the heuristic function, the imaginative function and the informative function. Nevertheless, the adult’s language becomes much more complex and it has to serve many more functions, and the original functional range of the child’s language is gradually reduced to a set of highly coded and abstract functions, which are meta-functions: the ideational, the interpersonal and the textual functions (Hu Zhuanglin, 2006: 311-312). Halliday views the meta-functional principle as the principle that “has shaped the organization of meaning in language; and (with trivial exceptions) every act of meaning embodies all three meta-functional components.” (Halliday, 2007: 18) Namely, meta-functions is the common feature of language.
...
	\subsection{Ideational Function}
	    ...
	\subsection{Interpersonal Function}
		...
	\subsection{Textual Function}
	...