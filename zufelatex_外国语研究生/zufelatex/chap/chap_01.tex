\chapter{Task Description}
\section{Background and Significance of the Task}
	Economy is a general term for production, distribution, and consumption in the society, intertwining with nearly every aspect of people’s life. Economics research has long been one of essential issues that countries and regions shall and should put on the table while pursuing the national development and prosperity. In this regard, the annual Nobel Prize in Economic Sciences has drawn much attention from a lot of scholars and experts in various realms worldwide for its authority, credibility and perspectiveness.
	...
\section{Introduction of the Source Text and Its Author}
	The 2015 Nobel Prize in Economic Sciences goes to Deaton for his unparalleled contributions to economics measuring, consumption behavior, social welfare and global poverty. Based upon mass diachronic household investigation and econometric models, the source text mainly discusses the “abnormal” phenomena of poverty in some regions, and gives explanations to the relation between inconsistent measuring methods and economic equality or inequality. Also, the source text deals with Deaton’s research conclusions on both theoretical and practical applications of economics.
	...

\section{Previous Studies on Economics Text Translation}
	After reading and classifying loads of available materials, the previous studies on economics text translation are conducted mainly in the following two perspectives: perspective of functional theory as well as perspective of linguistic strategies.
	\subsection{Perspective of Functional Theory}
	From a functional view, Chifane (2012) throws light on economic features and the “equivalence” effect in the course of English-Romanian translation and Romanian-English translation, focusing upon the problems emerging from the lack of lexical equivalence with examples. Consequently, the research points out that non-equivalence issues can be successfully addressed by means of employing appropriate translation strategies, such as paraphrase, explanation, cultural substitution word, omission and so on. Base on the Chinese-English translation of Chinese Economy at the Crossroads (excerpt), Pan Lingling (潘玲玲,2013) highlights the unique characteristics of economics text, specifically the expertise and normalization. She also agrees with Chifane over the forementioned translation methods to realize the final equivalence, and lays much more stress on the preciseness of terminology translation as well as translator’s preparation, attitudes and responsibility.
	...
	
	To sum up, with an outpouring of related researches in the recent decade, the present author thinks that although lots of the previous studies on the translation of economics texts are rather systematic and impressive, there is still much to be explored in this field. The coverage of translation language and guiding theory involved in economics text rendering relatively falls short of both expectation and demand. Great efforts should be cast in multilingual researches on economics translation practice from more diverse perspectives, which conceivably, will be likely to meet multifarious needs of the translation community and other potential users. In addition, employing corpus, eye-tracking and key logging data in the researches on economics text translation is indeed a remarkable and promising step, however, the focal point in some reports still merely lies in the version analysis of translation non-equivalents, misinterpretations or translationese.