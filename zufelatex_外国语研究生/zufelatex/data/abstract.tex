
% 下面是中文摘要 3-19行不用管
\setstretch{2}
\vspace*{2em}

\begin{center}
	\zihao{3}\textbf{ABSTRACT}
	\vspace{2mm}
	\ifx\subtitle\undefined\else
	%如果有副标题
	\begin{spacing}{1.2}
		\sanhao\selectfont{\textmd{\kaishu{-----}\subtitle}}
	\end{spacing}
	\fi
\end{center}
%题目后空一行
\vspace{1em}
\setstretch{1.5}

\setlength{\parskip}{0em}

% 英文摘要正文从这里开始--------------------------------------------------
As economic globalization continues to advance in depth, the translation of academic literature on economics has increasingly demonstrated its significance. This report selects and translates the 2015 Nobel Prize laureate’s academic literature into Chinese. Based on the research of the selected text and its Chinese translation by combining the case analysis with the comparative analysis, the report attempts to focus on what translation strategies are more effective in terms of realizing the ideational, interpersonal, and textual functions of the source text. The selected materials of this translation practice report do not have any existing translation versions for reference and the present author uses computer-aided technology to conduct a data analysis of the full original text before translation so as to learn about the features of the original.
\par
This report is guided by the three meta-functions of Halliday’s systemic functional linguistics. Through the analysis, it is concluded that annotation of terminologies, division of long sentences, and conversion of passive voice can achieve the ideational functions of the translation; addition of personal pronouns, literal translation of modal verbs and substitution of mood adjuncts can realize the interpersonal function; omission of connectives and conjunctive adjuncts, reproduction of demonstrative pronouns and adverbs, and reorganization of sentence order and paragraphs can achieve the textual function. The practicability of these strategies has been confirmed by concrete examples in this report. For the translation of academic literature on economics, comprehension and expression are pivotal. The translator should be familiar with the characteristics and style of the text, proficient in translation skills with the help of computer-aided tools, and flexible in dealing with rendering difficulties. This report is also intended to give some implications to both the translation practice of and theoretical application to the similar texts in the future.
\par

\vspace{1em}
\textbf{\heiti\xiaosi{Key words:}}
%这里写你的关键词------------------------------------------------------
academic literature on economics; three meta-functions of language; translation strategies

\newpage
\setstretch{2}
\vspace*{2em}
%如果没有英文标题,跳过
%下面是英文摘要 37-52行不用管
\begin{spacing}{0.95}
	\centering
	\sanhao\textbf{\heiti{摘要}}
	\vspace{2mm}
	\ifx\subtitle\undefined\else
	%如果有副标题
	\begin{spacing}{1.2}
		\sihao\selectfont{\textmd{\kaishu{-----}\rmfamily\textbf{\ensubtitle}}}
	\end{spacing}
	\fi
\end{spacing}
\vspace{1em}
\setstretch{1.5}
\setlength{\parskip}{0em}

% 中文摘要正文从这里开始--------------------------------------------------
随着经济全球化不断纵深推进,经济学类学术文献的翻译愈发显示出其重要性。本报告选择2015年诺贝尔经济学奖获得者的学术讲演稿进行汉译,基于对所选文本及其汉译文的研究,结合个案分析法与对比分析法,试图重点考察经济学类学术文献的翻译过程中采用何种策略可以在译文中更为有效地实现概念功能、人际功能、语篇功能等三大语言纯理功能。本翻译实践报告的所选材料无任何现有译文可供参考,笔者为了解原文特点,在译前借助计算机辅助技术对全文进行了数据分析。
\par
本报告以韩礼德系统功能语言学理论的三大语言纯理功能为指导,通过分析,结论如下:对术语加注、拆分长句和转换被动语态能够实现译文的概念功能;增译人称代词、直译情态动词和替换语气附加语可以实现译文的人际功能;省却连词和连接附加语、再现指示代词和指示副词,以及重组句序和段落能够实现译文的语篇功能,这些策略的实用性已由本报告的翻译实例得到印证。对于经济学类学术文献的翻译,理解能力与表达方式至关重要,译者当熟悉文体特征,精于翻译技巧并辅以机辅工具,灵活应对翻译难点。本报告对今后类似文本的翻译实践及理论应用也具有一定的启示意义。
\par

\vspace{1em}
\textbf{\rmfamily\xiaosi{关键词:}}
%这里写你的英文关键词----------------------------------------------------
经济学类学术文献;三大语言纯理功能;翻译策略


